\section{Rings and Modules of Fractions}
The formation of rings of fractions and the associated process of localization are perhaps the most important technical tools in commutative algebra. They correspond in the algebro-geometric picture to concentrating attention on an open set or near a point, and the importance of these notions should be self-evident. This section gives the definitions and simple properties of the formation of fractions.
\subsection{Basic Concepts}
In this section we shall sketch a brief review of the fraction of a ring. Most results presented here have been introduced in Section 4.4.\par
Recall the construction of the ring of fractions as developed in Section 34.4. Let $A$ be any ring. A \textbf{multiplicative subset} $S$ of $A$ is defined to be a subset of $A$ such that $1\in S$, and $S$ is closed under multiplication. Therefore $S$ is indeed a sub-semigroup of $A$. Define a relation $\sim$ on $A\times S$ as follows: 
$$(a,s)\sim (b,t)\iff (at-bs)u=0\ \text{for some}\ u\in S.$$
Clearly this is an equivalent relation. Let $a/s$ denote the equivalence class of $(a,s)$ and let $S^{-1}A$ denote the set of equivalence classes. We put a ring structure on $S^{-1}A$ by defining addition and multiplication of these "fractions" $a/s$ in the same way as in elementary algebra.\par
The ring of fractions $S^{-1}A$ may be characterized by the following universal property. Let $g:A\to B$ be a ring homomorphism such that $g(s)$ is a unit in $B$ for all $s\in S$. Then there exists a unique ring homomorphism $h:S^{-1}A\to B$ such that $g=h\circ f$. Here is a Corollary of the universal property of the ring of fractions that may be useful: 
\begin{corollary}
If $g:A\to B$ is a ring homomorphism such that \par
(i) $s\in S$ implies $g(s)$ is a unit in $B$;\par
(ii) $g(a)=0$ implies $as=0$ for some $s\in S$;\par
(iii) Every element of $B$ is of the form $g(a)g(s)^{-1}$,\par
then there is a unique isomorphism $g:S^{-1}A\to B$ such that $g=h\circ f$.
\end{corollary}
\begin{proof}
By the universal property of $S^{-1}A$ we know that there exists a homomorphism $h:S^{-1}A\to B$ such that $g=g\circ f$. Now it suffices to show that $h$ is indeed an isomorphism. Note that $h\left( a/s \right) =g\left( a \right) g\left( s \right) ^{-1}$, we have $h$ is surjective by (iii). Now if $h(a/s)=0$, then $g(a)=0$, which implies $at=0$ for some $t\in S$ and hence $a/s=0$ in $S^{-1}A$.
\end{proof}
Now we offer some examples.
\begin{example}\em
(i) Let $\mathfrak{p}$ be a prime ideal of $A$. Then $S=A-\mathfrak{p}$ is a multiplicative subset. We write $A_\mathfrak{p}$ for $S^{-1}A$ in this case. The elements $a/s$ with $a\in\mathfrak{p}$ form an ideal $\mathfrak{m}$ in $A_\mathfrak{p}$. It is the unique maximal ideal of $A_\mathfrak{p}$; in other words, $A_\mathfrak{p}$ is a local ring. Such process of passing $A$ to $A_\mathfrak{p}$ is called \textbf{localization} at $\mathfrak{p}$.\par
(ii) $S^{-1}A$ is the zero ring if and only if $0\in S$. To see this, suppose $S^{-1}A$ is the zero ring, therefore $(a,s)\sim(0,t)$ for some $t\in S$. Therefore $atu=0$ for some $u$ and hence $t=0$. Conversely, suppose $s=0\in S$, then for all $a\in A$ and $t\in S$ we have $(a,s)\sim (0,t)$ and hence $a/s=0$.\par
(iii) Let $f\in A$ and $S=\{f^n\}_{n\ge 0}$. Then $S$ is a multiplicative subset of $A$ and we shall denote $S^{-1}A$ as $A_f$.\par
(iv) Let $\mathfrak{a}$ be an ideal of $A$ and let $S=1+\mathfrak{a}$. Then $S$ is a multiplicative subset of $A$ and we may define $S^{-1}A$.\par
For some special cases, see\par
(v) Consider the ring $\mathbb{Z}$. $(p)$ is a prime ideal of $\mathbb{Z}$ if $p$ is a prime number. Then the ring $\mathbb{Z}_{(p)}$ is the ring of all fractions $m/n$ with $n$ prime to $p$. If $f\in\mathbb{Z}$ and $f\ne 0$, then $\mathbb{Z}_f$ is the set of all rational numbers whose denominator is a power of $f$.\par
(vi) Consider the ring $k[t_1,\cdots,t_n]$, where $k$ is a field. Suppose $\mathfrak{p}$ is a prime ideal of $k$, then if we denote $A=k[t_1,\cdots,t_n]$, $A_\mathfrak{p}$ is the ring consists of all rational functions $f/g$ such that $g\notin\mathfrak{p}$. Let $V=\{x=(x_1,\cdots,x_n)\in k^n: f(x)=0, f\in\mathfrak{p}\}$ be the variety defined by the prime ideal $\mathfrak{p}$, then (provided $k$ is infinite) $A\mathfrak{p}$ can be identified with the ring of all rational functions on $k^n$ which are defined at almost all points of $V$; it is the local ring of $k^n$ \textbf{along the variety $V$}. This is the prototype of the local rings which arise in algebraic geometry.
\end{example}
The construction of $S^{-1}A$ may be carried through with an $A$-module $M$ in place of the ring $A$. Define a relation $\sim$ on $M\times S$ as follows: 
$$(m,s)\sim (m^\prime,s^\prime)\iff t(sm^\prime-s^\prime m)=0\ \text{for some}\ t\in S.$$
As before, this is an equivalence relation. Let $m/s$ denote the equivalence class of the pair $(m,s)$, let $S^{-1}M$ be the set of such fractions, then we may make $S^{-1}M$ into an $S^{-1}A$-module with obvious definitions of addition and scalar multiplication. As in Example 8.1 (i) and (iii), we shall write $M_\mathfrak{p}$ when $S=A-\mathfrak{p}$ and $M_f$ when $S=\{f^n\}_{n\ge 0}$.\par
Let $u:M\to N$ be a homomorphism of $A$-modules. Then it give rise to an $S^{-1}A$-module homomorphism $S^{-1}u\to S^{-1}M\to S^{-1}N$, namely $S^{-1}u$ maps $m/s$ to $u(m)/s$. We have immediately by definition that $S^{-1}(N+P)=S^{-1}N+S^{-1}P$, $S^{-1}(N\cap P)=(S^{-1}N)\cap(S^{-1}P)$, and $S^{-1}(u\circ v)=(S^{-1}u)\circ(S^{-1}v)$.
\begin{proposition}
The operation $S^{-1}$ is exact, i.e. if $M^{\prime}\overset{f}{\longrightarrow}M\overset{g}{\longrightarrow}M^{\prime\prime}$ is exact at $M$, then $S^{-1}M^{\prime}\overset{S^{-1}f}{\longrightarrow}S^{-1}M\overset{S^{-1}g}{\longrightarrow}S^{-1}M^{\prime\prime}$ is exact at $S^{-1}M$.
\end{proposition}
\begin{proof}
First note that $g\circ f=0$. Therefore $s^{-1}(g\circ f)=(S^{-1}g)\circ(S^{-1}f)=0$ and hence $\mathrm{Im}(S^{-1}f)\subset\mathrm{Ker}(S^{-1}g)$. To prove the reverse inclusion, let $m/s\in\mathrm{Ker}(S^{-1}g)$, then $g(m/s)=g(m)/s=0$, hence there exists some $t\in S$ such that $tg(m)=g(tm)=0$, whence $tm\in\mathrm{Ker}g=\mathrm{Im}f$. Suppose $f(m^\prime)=tm$, consider $S^{-1}f(m^\prime/ts)=f(m^\prime)/ts=tm/ts=m/s$, hence $m/s\in\mathrm{Im}(S^{-1}g)$, which finished the proof.
\end{proof}
In particular, it follows that if $M^\prime$ is a submodule of $M$, the mapping $S^{-1}M^\prime\to S^{-1}M$ is injective and therefore $S^{-1}M^\prime$ can be regarded as a submodule of $S^{-1}M$.
\begin{corollary}
Suppose $M$ is an $A$-module and $N$ a submodule of $M$, then we have the isomorphism $S^{-1}(M/N)\cong(S^{-1}M)/(S^{-1}N)$.
\end{corollary}
\begin{proof}
Consider the short exact sequence 
$$
0\longrightarrow N\overset{\iota}{\longrightarrow}M\overset{\pi}{\longrightarrow}M/N\longrightarrow 0,
$$
apply $S^{-1}$ operator to obtain 
$$
0\longrightarrow S^{-1}N\overset{S^{-1}\iota}{\longrightarrow}S^{-1}M\overset{S^{-1}\pi}{\longrightarrow}S^{-1}\left( M/N \right) \longrightarrow 0,
$$
note that $\mathrm{Ker}(S^{-1}\pi)=\mathrm{Im}(S^{-1}\iota)=S^{-1}N$, we therefore have 
$$
S^{-1}\left( M/N \right) \cong \left( S^{-1}M \right) /\mathrm{Ker}\left( S^{-1}\pi \right) =\left( S^{-1}M \right) /\mathrm{Im}\left( S^{-1}\iota \right) \cong \left( S^{-1}M \right) /\left( S^{-1}N \right) ,
$$
which finished the proof.
\end{proof}
\begin{proposition}
Let $M$ be an $A$-module. Then the $S^{-1}$-modules $S^{-1}M$ and $S^{-1}A\otimes_AM$ are isomorphic under the isomorphism $f:S^{-1}A\otimes_AM\to S^{-1}M$ given by $(a/s)\otimes m\mapsto am/s$ for all $a\in A$, $m\in M$ and $s\in S$.
\end{proposition}
\begin{proof}
We first show that $f$ is well-defined. Indeed, observe that the map $((a/s),m)\mapsto am/s$ is a bilinear map, we therefore conclude that $f$ is well-defined by the universal property of tensor products. Trivially $f$ is surjective, we now show that $f$ is injective. First we show that every element in $S^{-1}A\otimes_AM$ is of the form $(1/s)\otimes m$. To see this, suppose $\sum_i(a_i/s_i)\otimes m_i$ be any element of $S^{-1}A\otimes_AM$. If $s=\prod_is_i\in S$, $t=\prod_{j\ne i}s_j$, we have 
$$
\sum_i{\frac{a_i}{s_i}\otimes m_i}=\sum_i{\frac{a_it_i}{s}\otimes m_i}=\sum_i{\frac{1}{s}\otimes a_it_im_i}=\frac{1}{s}\otimes \sum_i{a_it_im_i}=\frac{1}{s}\otimes m.
$$
Now suppose $(1/s)\otimes m\in\mathrm{Ker}f$. Then $m/s=0$, hence there exists some $t\in S$ such that $mt=0$, and hence 
$$
\frac{1}{s}\otimes m=\frac{t}{st}\otimes m=\frac{1}{st}\otimes tm=\frac{1}{st}\otimes 0=0,
$$
whence $(1/s)\otimes m=0$ and $f$ is an isomorphism.
\end{proof}
\begin{corollary}
$S^{-1}A$ is a flat $A$-module.
\end{corollary}
\begin{proof}
Suppose $0\longrightarrow M^{\prime}\longrightarrow M\longrightarrow M^{\prime\prime}\longrightarrow 0$ is a short exact sequence of $A$-modules, then $0\longrightarrow S^{-1}M^{\prime}\longrightarrow S^{-1}M\longrightarrow S^{-1}M^{\prime\prime}\longrightarrow 0$ is also a short exact sequence by Proposition 8.2. However by Proposition 8.4 we have the short exact sequence is equivalent to 
$$
0\longrightarrow S^{-1}A\otimes M^{\prime}\longrightarrow S^{-1}A\otimes M\longrightarrow S^{-1}A\otimes M^{\prime\prime}\longrightarrow 0
$$
which implies $S^{-1}A$ a flat $A$-module.
\end{proof}
\begin{proposition}
If $M$ and $N$ are $A$-modules, there is a unique isomorphism of $S^{-1}A$-modules $f:S^{-1}M\otimes_{S^{-1}A}S^{-1}N\to S^{-1}(M\otimes N)$ such that $(m/s)\otimes(n/t)\mapsto(m\otimes n)/st$. In particular, if $\mathfrak{p}$ is a prime ideal, then 
$$
M_{\mathfrak{p}}\otimes _{A_{\mathfrak{p}}}N_{\mathfrak{p}}\cong \left( M\otimes _AN \right) _{\mathfrak{p}}.
$$
\end{proposition}
\begin{proof}
We observe by Proposition 8.4 that 
$$
\begin{aligned}
S^{-1}M\otimes _{S^{-1}A}S^{-1}N&\cong \left( S^{-1}A\otimes _AM \right) \otimes _{S^{-1}A}\left( S^{-1}A\otimes _AN \right) =S^{-1}A\otimes _A\left( M\otimes _{S^{-1}A}S^{-1}A \right) \otimes _AN
\\
&\cong S^{-1}A\otimes _AM\otimes _AN=S^{-1}A\otimes _A\left( M\otimes _AN \right) =S^{-1}\left( M\otimes _AN \right) ,
\end{aligned}
$$
which finished the proof.
\end{proof}